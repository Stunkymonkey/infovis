\documentclass[a4paper]{article}
\usepackage[utf8]{inputenc}
\usepackage{textcomp}
\usepackage{geometry}
\geometry{ left=2cm, right=2cm, top=2cm, bottom=4cm, bindingoffset=5mm}
\usepackage{graphicx}
\usepackage{xcolor}
\usepackage{hyperref}
\date{}
\author{}
\usepackage{fancyhdr}
\pagestyle{fancy}
\fancyhf{}
\fancyhead[R]{ 2973140 - Felix Bühler \\ 2892258 - Gerhard Breul \\  3141241 - Jamie Ullerich}
\fancyhead[L]{Information Visualisation and Visual Analytics \\ WS 2019/20 }
\renewcommand{\headrulewidth}{0.5pt}
\usepackage{tikz}
\usetikzlibrary{calc}
\usepackage{amsmath}
\usepackage{cleveref}
\usepackage{subcaption}

\usepackage{changepage,lipsum,titlesec}
\titleformat{\section}[block]{\bfseries}{\thesection.}{1em}{}
\titleformat{\subsection}[block]{}{\thesubsection}{1em}{}
\titleformat{\subsubsection}[block]{}{\thesubsubsection}{1em}{}
\titlespacing*{\subsection} {2em}{3.25ex plus 1ex minus .2ex}{1.5ex plus .2ex}
\titlespacing*{\subsubsection} {3em}{3.25ex plus 1ex minus .2ex}{1.5ex plus .2ex}


\title{\textbf{Assignment 7}}

\begin{document}
\maketitle 
\thispagestyle{fancy}

\section*{Task 1 - The Information Visualization Reference Model}
\begin{enumerate}
	\item[(a)]
		\begin{enumerate}
			\item[i.]
			\item[ii.]
			\item[iii.]
		\end{enumerate}
	\item[(b)]
	\item[(c)]
	\item[(d)]
\end{enumerate}


\section*{Task 2 - Interaction with Interactive Visualizations}
\begin{enumerate}
	\item[(a)]The user can select two nodes. The connecting edge will be displayed, as soon as two markers are selected.
	\item[(b)]Zooming in, but keeping the same sizes for each node. Thereby the nodes will split up eventually.
	\item[(c)]On the side there is a filter where you can select different colors. The nodes of the selected color are getting bigger and thereby easier to select.
\end{enumerate}

\section*{Task 3 - k-d Trees}
\begin{enumerate}
	\item[(a)] See \Cref*{kdtree} for the resulting tree.
	\begin{figure}
		\centering
		\begin{subfigure}[t]{.4\textwidth}
			\includegraphics[height=5cm]{2-dtree.pdf}
			\caption{Data points visualised as areas.}
		\end{subfigure}
	\begin{subfigure}[t]{.4\textwidth}
		\includegraphics[width=\linewidth]{2dtree_nodelink.pdf}
		\caption{Data points visualised as a node-link diagram.}
	\end{subfigure}
	\caption{k-d Tree}
	\label{kdtree}
	\end{figure}
	\item[(b)] The problem with this tree is, that it is not balanced. 
	This can be fixed by taking the median point, that is changing the order of the given points in task (a). 
	The resulting tree can be seen in \Cref*{kdtree2}.
	\begin{figure}
		\centering
	\begin{subfigure}[t]{.4\textwidth}
		\includegraphics[width=\textwidth]{kd-tree2.pdf}
		\caption{Data points visualised as areas.}
	\end{subfigure}
	\begin{subfigure}[t]{.4\textwidth}
		\includegraphics[width=\linewidth]{2dtree_nodelink2.pdf}
		\caption{Data points visualised as a node-link diagram.}
	\end{subfigure}
	\caption{balanced k-d Tree}
	\label{kdtree2}
\end{figure}
\end{enumerate}

\end{document}
