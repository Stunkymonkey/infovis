\documentclass[a4paper]{article}
\usepackage[ngerman]{babel}
\usepackage[T1]{fontenc}
\usepackage[utf8]{inputenc}
\usepackage{textcomp}
\usepackage{geometry}
\geometry{ left=2cm, right=2cm, top=2cm, bottom=4cm, bindingoffset=5mm}
\usepackage{graphicx}
\usepackage{xcolor}
\usepackage{hyperref}
\date{}
\author{}
\usepackage{fancyhdr}
\pagestyle{fancy}
\fancyhf{}
\fancyhead[R]{3141241 - Jamie Ullerich \\ 2892258 - Gerhard Breul \\ 2973140 - Felix Bühler}
\fancyhead[L]{Information Visualisation and Visual Analytics \\ WS 2019/20 }
\renewcommand{\headrulewidth}{0.5pt}
\usepackage{tikz}
\usetikzlibrary{calc}



\title{Assignment 2}

\usepackage[ngerman]{babel}
\usepackage[T1]{fontenc}
\usepackage[utf8]{inputenc}
\usepackage{textcomp}
\usepackage{geometry}
\geometry{ left=2cm, right=2cm, top=2cm, bottom=4cm, bindingoffset=5mm}
\usepackage{graphicx}
\usepackage{xcolor}
\usepackage{hyperref}
\date{}
\author{}
\usepackage{fancyhdr}
\pagestyle{fancy}
\fancyhf{}
\fancyhead[R]{3141241 - Jamie Ullerich \\ 2892258 - Gerhard Breul \\ 2973140 - Felix Bühler}
\fancyhead[L]{Information Visualisation and Visual Analytics \\ WS 2019/20 }
\renewcommand{\headrulewidth}{0.5pt}
\usepackage{tikz}
\usetikzlibrary{calc}



\title{Assignment 2}

\begin{document}
	\maketitle 
	\thispagestyle{fancy}
	
	
	\section*{Task 1 - Scales and Visual Mapping}
	\subsection*{a)}
	A ratio scale has a meaningful zero value, which makes multiplicative operations possible. This allows for statements about ratios such as 'value a is 2x greater than value b'. \\For interval scales the difference between values e.g. 10°C and 20°C is the same as the difference between 110°C and 120°C. The term interval might result from the ability to declare intervals and ratio might result from the ability to declare ratios between all values.
	\subsection*{b)}
	\begin{itemize}
		\item{\makebox[4cm]{degree Celsius\hfill} = interval}
		\item{\makebox[4cm]{Kelvin\hfill} = ratio}
		\item{\makebox[4cm]{dates\hfill} = interval}
		\item{\makebox[4cm]{durations\hfill} = ratio}
		\item{\makebox[4cm]{Cartesian coordinates\hfill} = interval}
		\item{\makebox[4cm]{weight\hfill} = ratio}
		\item{\makebox[4cm]{account balance\hfill} = ratio}
		\item{\makebox[4cm]{length\hfill} = ratio}
	\end{itemize}
	\subsection*{c)}
	\begin{itemize}
		\item Persons:
		\definecolor{c11}{RGB}{228,26,28}
		\definecolor{c12}{RGB}{55,126,184}
		\definecolor{c13}{RGB}{77,175,74}
		\definecolor{c14}{RGB}{152,78,163}
		\definecolor{c15}{RGB}{255,127,0}
		\definecolor{c16}{RGB}{255,255,51}
		\definecolor{c17}{RGB}{166,86,40}
		
\begin{figure}[!h]
		
	\centering
		
	\begin{tikzpicture}[remember picture]
		
	\node [rectangle, fill, color=c11, text=black, anchor=east, minimum width=2.2cm, minimum height=0.7cm] (a){Max};
		
	\node [rectangle, fill, color=c12, text=black, anchor=east, minimum width=2.2cm, minimum height=0.7cm, right of=a,node distance=2.4cm] (b){Ben};
		
	\node [rectangle, fill, color=c13, text=black, anchor=east, minimum width=2.2cm, minimum height=0.7cm, right of=b,node distance=2.4cm] (c){Alice};
		
	\node [rectangle, fill, color=c14, text=black, anchor=east, minimum width=2.2cm, minimum height=0.7cm, right of=c,node distance=2.4cm] (d){Bob};
		
	\node [rectangle, fill, color=c15, text=black, anchor=east, minimum width=2.2cm, minimum height=0.7cm, right of=d,node distance=2.4cm] (e){Johanna};
		
	\node [rectangle, fill, color=c16, text=black, anchor=east, minimum width=2.2cm, minimum height=0.7cm, right of=e,node distance=2.4cm] (f){Tiffany};
		
	\node [rectangle, fill, color=c17, text=black, anchor=east, minimum width=2.2cm, minimum height=0.7cm, right of=f,node distance=2.4cm] (g){Andrea};
		
	\end{tikzpicture}
		
\end{figure}
		We wanted to concretely distinguish between all the names, therefore we used 7 different colors.
		
		\item Grades:
		\definecolor{c21}{RGB}{26,152,80}
		\definecolor{c22}{RGB}{145,207,96}
		\definecolor{c23}{RGB}{217,239,139}
		\definecolor{c24}{RGB}{254,224,139}
		\definecolor{c25}{RGB}{252,141,89}
		\definecolor{c26}{RGB}{215,48,39}
		
		\begin{figure}[!h]
			\centering
			\begin{tikzpicture}[remember picture]
			\node [rectangle, fill, color=c21, text=black, anchor=east, minimum width=2.4cm, minimum height=0.7cm] (a){very good};
			\node [rectangle, fill, color=c22, text=black, anchor=east, minimum width=2.4cm, minimum height=0.7cm, right of=a,node distance=2.6cm] (b){satisfactory};
			\node [rectangle, fill, color=c23, text=black, anchor=east, minimum width=2.4cm, minimum height=0.7cm, right of=b,node distance=2.6cm] (c){good};
			\node [rectangle, fill, color=c24, text=black, anchor=east, minimum width=2.4cm, minimum height=0.7cm, right of=c,node distance=2.6cm] (d){sufficientient};
			\node [rectangle, fill, color=c25, text=black, anchor=east, minimum width=2.4cm, minimum height=0.7cm, right of=d,node distance=2.6cm] (e){unsatisfactory};
			\node [rectangle, fill, color=c26, text=black, anchor=east, minimum width=2.4cm, minimum height=0.7cm, right of=e,node distance=2.6cm] (f){poor};
			\end{tikzpicture}
		\end{figure}
		Usually green indicates something positive ('go' on a traffic light). Red instead indicates something negative such as danger ('stop' on a traffic light). Therefore we used green for good things and red for bad things.
		\item Degree Celsius:
		
		\begin{figure}[!h]
			\centering
			\begin{tikzpicture}[remember picture]
			\node [rectangle, left color=blue, right color=red, anchor=south, minimum width=\textwidth-2cm, minimum height=0.7cm] {};

			\node [rectangle, anchor=north, minimum height=0.7cm] at (-0.0 / 70* \textwidth, 0) {-0.0};
			\draw (-0.0 / 70* \textwidth,-0.1) -- (-0.0 / 70* \textwidth,0.7);
			\node [rectangle, anchor=north, minimum height=0.7cm] at (-3.5 / 70 * \textwidth, 0) {-3.5};
			\draw (-3.5 / 70 * \textwidth,-0.1) -- (-3.5 / 70* \textwidth,0.7);
			\node [rectangle, anchor=north, minimum height=0.7cm] at (2.1 / 70 * \textwidth, 0) {+2.1};
			\draw (2.1 / 70 * \textwidth,-0.1) -- (2.1 / 70* \textwidth,0.7);
			\node [rectangle, anchor=north, minimum height=0.7cm] at (13.5 / 70 * \textwidth, 0) {+13.5};
			\draw (13.5 / 70* \textwidth,-0.1) -- (13.5 / 70* \textwidth,0.7);
			\node [rectangle, anchor=north, minimum height=0.7cm] at (5.2 / 70* \textwidth, 0) {+5.2};
			\draw (5.2 / 70* \textwidth,-0.1) -- (5.2 / 70* \textwidth,0.7);
			\node [rectangle, anchor=north, minimum height=0.7cm] at (-6.0 / 70* \textwidth, 0) {-6.0};
			\draw (-6.0 / 70* \textwidth,-0.1) -- (-6.0 / 70* \textwidth,0.7);
			\node [rectangle, anchor=north, minimum height=0.7cm] at (22.7 / 70* \textwidth, 0) {+22.7};
			\draw (22.7 / 70* \textwidth,-0.1) -- (22.7 / 70* \textwidth,0.7);
			\node [rectangle, anchor=north, minimum height=0.7cm] at (18.3 / 70* \textwidth, 0) {+18.3};
			\draw (18.3 / 70* \textwidth,-0.1) -- (18.3 / 70* \textwidth,0.7);

			\end{tikzpicture}
		\end{figure}
		Red usually means hot and blue means cold. therefore value zero is a mixture and in our case purple.
		
	\end{itemize}
	\subsection*{d)}
	
	visual variables:
	\begin{itemize}
		\item bar chart:
		\begin{itemize}
			\item lenght
			\item position
		\end{itemize}
		\item pie chart:
		\begin{itemize}
			\item angle
			\item area
		\end{itemize}
	\end{itemize}
	When sorting values in increasing order, the bar chart has an advantage over the pie chart. With respect to the visual variables it is easier to interpret and detect differences in length than in angle and area. If the task was to express something as a percentage the pie chart would make the data more easily understandable.
	
	
	\section*{Task 2 - Storytelling}
	
	\includegraphics[width=\linewidth]{climatechangevis.pdf}
		\begin{itemize}		
				\item Comprehensibility: The visual representation of the relation between temperature, Sea level and atmospheric CO2 content over a period of 50 years makes it clear that rising CO2-levels are related to warmer climate and rising sea levels. The causes can be understood at a glance thanks to the pictures and are elaborated on more closely by the text next to each of them.
				\item Likability: Thanks to the scales' colorfulness and drawings depicting the causes, the visualization becomes more visually interesting and engaging.
				\item To improve navigability, dashed lines separate the visualization into three parts. This helps to separate the part regarding the causes of climate change form the part regarding its consequences and structures the visualization.

		\end{itemize}

%	
%	\newpage
%	
%	\begin{figure}[htb]
%		\rotatebox{90}{%	
%			\includegraphics[width=25cm]{climatechangevis.pdf}		
%		}
%	\end{figure}
\end{document}
