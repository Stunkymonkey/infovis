\documentclass[a4paper]{article}
\usepackage[ngerman]{babel}
\usepackage[T1]{fontenc}
\usepackage[utf8]{inputenc}
\usepackage{textcomp}
\usepackage{geometry}
\geometry{ left=2cm, right=2cm, top=2cm, bottom=4cm, bindingoffset=5mm}
\usepackage{graphicx}
\usepackage{xcolor}
\usepackage{hyperref}
\date{}
\author{}
\usepackage{fancyhdr}
\pagestyle{fancy}
\fancyhf{}
\fancyhead[R]{3141241 - Jamie Ullerich \\ 2892258 - Gerhard Breul \\ 2973140 - Felix Bühler}
\fancyhead[L]{Information Visualisation and Visual Analytics \\ WS 2019/20 }
\renewcommand{\headrulewidth}{0.5pt}
\usepackage{tikz}
\usetikzlibrary{calc}



\title{\textbf{Assignment 3}}

\begin{document}
\maketitle 
\thispagestyle{fancy}

\section*{Task 1 - Lie Factor}
\subsection*{1.1}
\subsection*{1.2}
\section*{Task 2 - Gestalt Principles}

\begin{enumerate}
	\item[a)]
	Law of Continuity: it is possible to read the two words ''Coca'' and ''Cola'', even though they overlap. 
	Law of Connectedness: the two words are connected with lines and therefore it is obvious which letters from the word. 
	\item[b)]
	Law of Closure: there is no outline in this logo, but we can read the letters IBM. The vertical lines are ending where the outline of the letter would be and therefore it seems as if the contours are closed.  
	\item[c)] 
	Law of Similarity: the rectangle is perceived as one group since it is made of the same objects which share the same features. 
	It is also possible to read the word ''SUN'' on the edges of the rectangle, even though it consists only of the letter ''U''.
	\item[d)] 
	Law of Proximity: the U is composed of different shapes and smaller objects, but it is still clear, that this forms the letter U. 
\end{enumerate}

\section*{Task 3 - Pre-Attentive Processing}
	
\end{document}
