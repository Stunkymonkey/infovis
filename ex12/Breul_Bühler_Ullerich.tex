\documentclass[a4paper]{article}
\usepackage[utf8]{inputenc}
\usepackage{textcomp}
\usepackage{geometry}
\geometry{ left=2cm, right=2cm, top=2cm, bottom=2cm, bindingoffset=5mm}
\usepackage{graphicx}
\usepackage{xcolor}
\usepackage{hyperref}
\date{}
\author{}
\usepackage{fancyhdr}
\pagestyle{fancy}
\fancyhf{}
\fancyhead[R]{2973140 - Felix Bühler \\ 2892258 - Gerhard Breul \\  3141241 - Jamie Ullerich}
\fancyhead[L]{Information Visualisation and Visual Analytics \\ WS 2019/20 }
\renewcommand{\headrulewidth}{0.5pt}
\usepackage{tikz}
\usetikzlibrary{calc}
\usepackage{amsmath}
\usepackage{cleveref}
\usepackage{subcaption}

\usepackage{changepage,titlesec}
\titleformat{\section}[block]{\bfseries}{\thesection.}{1em}{}
\titleformat{\subsection}[block]{}{\thesubsection}{1em}{}
\titleformat{\subsubsection}[block]{}{\thesubsubsection}{1em}{}
\titlespacing*{\subsection} {2em}{3.25ex plus 1ex minus .2ex}{1.5ex plus .2ex}
\titlespacing*{\subsubsection} {3em}{3.25ex plus 1ex minus .2ex}{1.5ex plus .2ex}
\setcounter{MaxMatrixCols}{20}

\title{\textbf{Assignment 12}}

\begin{document}
\maketitle 
\thispagestyle{fancy}

\section*{Task 1 - Time Visualization}
\begin{enumerate}
	\item[(a)] Time-to-time mapping would be great for visualising weather data, like for instance the prediction of the path a hurricane will probably take. 
	This would help people to prepare if they are at risk. 
	It would be possible to show a window, which contains the current state of the hurricane, illustrated as a weather map. 
	Then the user could press play or pause and see how the hurricane moves. 
	It might also be helpful to provide some kind of more accurate control over the time, like single frames or different speeds. 
	This is helps everyone to understand if they are in danger since this is a simple way to encode time. 
	\item[(b)] A Gnatt chart is a good example for time to space mapping. 
	Here the timeline at the top or bottom of the visualisation shows when different parts of a project should be finished. 
	The user cannot interact directly with the temporal dimension, unless there is a possibility to zoom in for instance, then the user could change the scale of the time to see either more details or a broad overview over a year. 
	\item[(c)] It is very useful, since it is possible to show more data then non-interactive visualisations. 
	For every time step, all information can be shown, which is sometimes very important like in the example mentioned in (a). 
	A weakness is, that it is requires more time to look at it since it cannot be shown in one visualisation. 
	Furthermore, it is better to have a digital version, since printing out would require to print every single time step. 
	\item[(c)] A weakness is, that one dimension must be kept free for the time and therefore cannot be used to illustrate other data. 
	This means it is not possible to illustrate stuff in as much detail as it is with time-to-time mapping. 
	But, it can be easily printed out, shown in papers or books and it does not require additional time to view the illustration. 
\end{enumerate}

\section*{Task 2 - Research: Horizon Graphs}

\begin{enumerate}
	\item[(a)] The researchers compared how well different types and sizes of charts convey time series information with regard to accuracy and speed. They found that multi-layered charts generally perform worse than those with fewer layers , but have an edge when chart size becomes very small.
	
	\item[(b)] The horizon graph conveys the same information line charts do, but displays negative values as hanging from the top of the chart as opposed to using the same scale as positive values.
	
	\item[(c)] Strenght: negative and positive values are visually very disinct. 
	\\Weakness: unusual approach which most people are unfamiliar with and necessitates a second scale.
	
	\item[(d)] Hypotheses: 
	\begin{enumerate}
		\item Offset graphs result in faster, more accurate comparisons than mirror graphs.
		\item Increasing the number of bands increases estimation time and decreases accuracy across graph variants.
		\item For larger chart heights, line charts perform better than mirror charts.
		\item 2-band horizon graphs result in better accuracy than the other chart types once the chart height falls under a threshold size.
	\end{enumerate}
The first experiment was conducted with 18 participants, the second one with 30 (plus eight more for the follow-up). They concluded that their first hypothesis was incorrect and that the third one was only partially correct. The other hypotheses where confirmed by the experiments' results.
	
\end{enumerate}

\end{document}
