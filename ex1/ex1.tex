\documentclass[a4paper]{article}
\usepackage[ngerman]{babel}
\usepackage[T1]{fontenc}
\usepackage[utf8]{inputenc}
\usepackage{textcomp}
\usepackage{geometry}
\geometry{ left=2cm, right=2cm, top=2cm, bottom=4cm, bindingoffset=5mm}
\usepackage{graphicx}
\usepackage{xcolor}
\usepackage{hyperref}
\date{}
\author{}
\usepackage{fancyhdr}
\pagestyle{fancy}
\fancyhf{}
\fancyhead[R]{3141241 - Jamie Ullerich \\ xxxxxxx - Gerhard Breul \\ 2973140 - Felix Bühler}
\fancyhead[L]{Information Visualisation and Visual Analytics \\ WS 2019/20 }
\renewcommand{\headrulewidth}{0.5pt}

\title{Assignment 1}

\begin{document}
	\maketitle 
	\thispagestyle{fancy}
	
	\section*{Task 1 - Differences in visualization disciplines}
	
	Scientific visualisation focuses on 3D dimensional visualisations, like renderings of volumes, surface and isolines. 
	These should be realistic, since it is mostly used in medical applications, for biological or metrological reasons. 
	Additionally, there can be a time component and the layout is mostly given (or self explanatory).\\ \linebreak
	On the other hand, information visualisation focuses on abstract data, like networks, documents or statistics. 
	Since there is no natural analogy, the spatial layout must be chosen, in contrast to scientific applications. 
	Therefore the layout needs more explanation. 
	
	
	\section*{Task 2 - Historical Visualization}
	
	\section*{Task 3 (Bonus Task) - Tag Cloud}
\end{document}
